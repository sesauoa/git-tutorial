\documentclass{tufte-handout}
\usepackage{listings}
\usepackage{cprotect}
\usepackage{multicol}
\usepackage{graphicx}

\definecolor{dkgreen}{rgb}{0,0.6,0}
\definecolor{gray}{rgb}{0.5,0.5,0.5}
\definecolor{mauve}{rgb}{0.58,0,0.82}

\lstset{language=bash,
  aboveskip=3mm,
  belowskip=3mm,
  showstringspaces=false,
  columns=flexible,
  basicstyle={\small\ttfamily},
  numbers=none,
  numberstyle=\tiny\color{gray},
  keywordstyle=\color{blue},
  commentstyle=\color{dkgreen},
  stringstyle=\color{mauve},
  breaklines=true,
  breakatwhitespace=true,
  tabsize=4,
  upquote=true
}

\title{Git Working!}
\author{SESA}

\begin{document}

\maketitle

\begin{abstract}
\noindent In this tutorial, we'll be covering the basics of git so that you'll
be able to use it when working on assignments throughout your degree. We'll be
assuming you're on a lab computer running Ubuntu, but if you've gotten git 
working on a personal device that you are more comfortable with, use that. It's
recommended that you type out all commands provided rather than copying and
pasting them. This is to help you become more familiar with a terminal and
get super fast at typing stuff out.
\end{abstract}

\noindent To being with, you're going to want to open up a terminal window and your
favourite text editor.

\section{Gitting Started}
\subsection{Initial Setup}

Before we do anything, let's get git setup.

\begin{lstlisting}
$ git config --global user.username "Your Name"
$ git config --global user.email "your_email@example.com"
\end{lstlisting}

\subsection{First Steps}

First, you're going to want to move into your workspace. Then, create a new 
directory for the repo. Finally, initialise the git repository.

\begin{lstlisting}
$ cd ~/workspace
$ mkdir git-tutorial
$ cd git-tutorial
$ git init
$ git status
\end{lstlisting}

\noindent Note down what's changed in the directory, what does git status show?

\subsection{Making History}
We'll need to make something into the repo for it to be useful at all. Make
anything you want really, but here's something\cprotect\footnote{
	\begin{lstlisting}[language=Python]
	#! /usr/bin/env python

	print 'Hello, world!'
	\end{lstlisting}
}. Add the file(s) you created to the repo and save it's state with a commit. 
It's a good idea to check the state of the repo along the way.

\begin{lstlisting}
$ touch hello_world.py
$ vim hello_world.py # Use you're favourite editor here.
$ python hello_world.py
Hello, world!
$ git status
$ git add .
$ git status
$ git commit -m 'hello world'
$ git status
\end{lstlisting}

What's printed out by git status? What is an untracked file? Do you know what
the . means when adding files? What does the -m option of git commit do?

\section{Branching Out}
\noindent It's time to add an `experimental' feature to our code. We don't
want to break master if we make a mistake so we'll be working on a different 
branch.

\subsection{Creating a Branch}

\noindent Let's create a branch for experimenting on. We'll call it develop as this is 
the industry standard.

\begin{lstlisting}
$ git branch develop
$ git branch
$ git checkout develop
$ git status
\end{lstlisting}

\noindent Notice what the current branch is from the output of git branch. Can
you tell what branch you're on using git status?

\subsection{More Changes}

\noindent Time to add that `experimental' feature. Go ahead and change the code to
produce something different \cprotect\footnote{
	\begin{lstlisting}[language=Python]
	#! /usr/bin/env python

	print 'Hello, world'
	print 'Hello, Nasser!'
	\end{lstlisting}
}.

\begin{lstlisting}
$ vim hello_world.py
$ python hello_world.py
Hello, world!
Hello, Nasser!
$ git status
$ git diff
$ git commit -am 'experimenting'
$ git status
\end{lstlisting}

\noindent What was shown by git diff, and do you understand what it means?

\subsection{Checking the Log}

\noindent Time to review what we've done to the repo. This is achieved by having a look
at the log that git keeps automatically.

\begin{lstlisting}
$ git log
\end{lstlisting}

\subsection{Merging changes}
\noindent Everything seems to be working fine so we haven't broken anything. 
Let's merge the changes we made back into the master branch.

\begin{lstlisting}
$ git checkout master
$ git log
$ python hello_world.py
Hello, world!
$ git merge develop
$ python hello_world.py
Hello, world!
Hello, Nasser!
$ git log
\end{lstlisting}

\noindent Notice how after you checkout the master branch, changes you made in
the develop branch aren't there until you merge. What does Fast-forward mean in
the output of git merge?

\section{Github}

Time to get your code on the line! We're going to set up a remote repository
and push your changes up to it. Pick a friend or random stranger (or use my
Chris' repo \url{https://github.com/drpotato/git-tutorial})

\end{document}